\documentclass[draft,12pt]{article}
\usepackage{simplex}
\usepackage[round]{natbib}
\usepackage{smooshedbib}
\begin{document}

\section{A morphonological example}

Latin verb roots ending in a coronal stop take the \emph{-s-} perfect allomorph and many verbs have corresponding agent nominals in \emph{-sor}. \citet{Heslin1987} observes that root-final coronal is assibilated after short vowels, as in (\ref{sperfects}a), and deleted before long nuclei, as in (\ref{sperfects}b).

\begin{example}[Perfect passive participles in \emph{-s-us} and agents in \emph{-sor}] \label{sperfects}
\begin{tabular}{l l l l l l l}
a. & \emph{metere}   & `reap'    & \emph{messus}  & `harvested' & \emph{messor}  & `reaper'   \\
   & \emph{fodere}   & `dig'     & \emph{fossus}  & `dug'       & \emph{fossor}  & `digger'   \\
b. & \emph{plaudere} & `applaud' & \emph{plausus} & `applauded' & \emph{plausor} & `cheerer'  \\
   & \emph{lūdere}   & `play'    & \emph{lūsus}   & `played'    & \emph{lūsor}   & `player' \\
\end{tabular}
\end{example}

\section{A mathematical example}

\citet{Zipf1949} notes a linear relationship between log word frequency $r$ and log frequency $r$. A generalized form of this relationship, shown in (\ref{zipf}), is what is now known as Zipf's Law \citep[e.g.,]{Baroni2009}.

\begin{unlabeledexample} \label{zipf}
$\displaystyle f(C, \alpha) = \frac{C}{r^\alpha}$
\end{unlabeledexample}

\bibliography{\jobname}
\bibliographystyle{abbrvnat}
\end{document}
