\documentclass[draft,12pt]{article}
\usepackage{simplex}
\usepackage[round]{natbib}
\usepackage{smooshedbib}
\usepackage{fullpage}

\usepackage{mathspec}
\setmainfont[Mapping=tex-text]{Linux Libertine}
\setmathfont(Digits,Greek,Latin){Linux Libertine}

\begin{document}

\section{A morphonological example}

Latin verb roots ending in a coronal stop take the \emph{-s-} perfect allomorph and many verbs have corresponding agent nominals in \emph{-sor}. \citet{Heslin1987} observes that root-final coronal is assibilated after short vowels, as in (\ref{sperfects}a), and deleted before long nuclei, as in (\ref{sperfects}b).

\begin{example}[Perfect passive participles and agents]
\vskip\baselineskip
\begin{tabular}{l l l l l l l}
a. & \emph{metere}   & `reap'    & \emph{messus}  & `harvested' & \emph{messor}  & `reaper'   \\
   & \emph{fodere}   & `dig'     & \emph{fossus}  & `dug'       & \emph{fossor}  & `digger'   \\
b. & \emph{plaudere} & `applaud' & \emph{plausus} & `applauded' & \emph{plausor} & `cheerer'  \\
   & \emph{lūdere}   & `play'    & \emph{lūsus}   & `played'    & \emph{lūsor}   & `player' \\
\end{tabular}
\label{sperfects}
\end{example}

\section{A gloss example}

\citet{Albright2005} observes that word-final *\emph{-ōs} scans as heavy \emph{-ōr} in the fragments of Ennius. 
At first blush, this suggests that leveling began before \textsc{Pre-Liquid Shortening} was actuated, preserving the allomorphy-reduction hypothesis.
However, word-final consonants syllabify as the onsets of following vowel-initial words (\citealt{Allen1978}:127) and in all of \citeauthor{Albright2005}'s examples, such as the following, word-final \emph{r} is followed by a vowel.

\begin{example}[\textsc{Pre-Liquid Shortening} bled by external sandhi]
\gll \emph{clāmōr}=\emph{ad} \emph{cael-um}          \emph{uolu-e-nd-us}                                \emph{per}=\emph{aether-a}
     shout=to                heaven-\textsc{acc.sg.} roll-\textsc{t}-\textsc{fut.pass}-\textsc{nom.sg.} through=heaven-\textsc{acc.pl.}
\glt `a shout fit to roll up to heaven' (fragments of Ennius)
\glend
\end{example}

\section{A mathematical example}

\citet{Zipf1949} notes a linear relationship between log word frequency $r$ and log frequency $r$. A generalized form of this relationship, shown in (\ref{zipf}), is what is now known as Zipf's Law \citep[e.g.,][]{Baroni2009}.

\begin{unlabeledexample} \label{zipf}
$\displaystyle f(C, \alpha) = \frac{C}{r^\alpha}$
\end{unlabeledexample}

\bibliography{\jobname}
\bibliographystyle{abbrvnat}
\end{document}
