\documentclass[draft,12pt]{article}
\usepackage{simplex}
\usepackage[round]{natbib}
\usepackage{smooshedbib}
\usepackage{fullpage}
\usepackage{libertinus}

\begin{document}

\section{Examples}

The \texttt{example} environment produces a consecutively numbered example. A printed label must be provided as an argument; this is used to name, describe, and/or cite what follows, and is rendered with a trailing colon and a blank line.

For example, Latin verb roots ending in a coronal stop take the \emph{-s-} perfect allomorph and many verbs have corresponding agent nominals in \emph{-sor}. \citet{Heslin1987} observes that root-final coronal is assibilated after short vowels, as in (\ref{ex:perfects-assibilate}),
and deleted before long nuclei, as in (\ref{ex:perfects-delete}).

\begin{example}[Assibilating perfect passive participles and agents]
\label{ex:perfects-assibilate}
\begin{tabular}[H]{llllll}
\emph{metere} & `reap' & \emph{messus} & `harvested' & \emph{messor} & `reaper' \\
\emph{fodere} & `dig'  & \emph{fossus} & `dug'       & \emph{fossor} & `digger' \\
\end{tabular}
\end{example}

\begin{example}[Deleting perfect passive participles and agents]
\label{ex:perfects-delete}
\begin{tabular}[H]{llllll}
\emph{plaudere} & `applaud' & \emph{plausus} & `applauded' & \emph{plausor} & `cheerer'  \\
\emph{lūdere}   & `play'    & \emph{lūsus}   & `played'    & \emph{lūsor}   & `player' \\
\end{tabular}
\end{example}

This environment also works well with interlinear glosses, for which we borrow from \texttt{covington.sty}. The syntax is the same except that \verb|\glend| is now no-op and can be omitted.

\citet{Albright2005} observes that word-final *\emph{-ōs} scans as heavy \emph{-ōr} in the fragments of Ennius. At first blush, this suggests that leveling began before \textsc{Pre-Liquid Shortening} was actuated, preserving the allomorphy-reduction hypothesis. However, word-final consonants syllabify as the onsets of following vowel-initial words (\citealt{Allen1978}:127) and in all of \citeauthor{Albright2005}'s examples, such as the following, word-final \emph{r} is followed by a vowel.

\begin{example}[\textsc{Pre-Liquid Shortening} bled by external sandhi]
\gll \emph{clāmōr}=\emph{ad} \emph{cael-um}          \emph{uolu-e-nd-us}                                \emph{per}=\emph{aether-a}
     shout=to                heaven-\textsc{acc.sg.} roll-\textsc{t}-\textsc{fut.pass}-\textsc{nom.sg.} through=heaven-\textsc{acc.pl.}
\glt `a shout fit to roll up to heaven' (fragments of Ennius)
% Omitted---noop.
%\glend
\end{example}

\section{Short examples}

The \texttt{shortexample} environment is similar to the \texttt{example} environment except the label appears on the same line. It can be used for

\begin{itemize}
\item simple rules that fit on a single line, or
\item lists of words and glosses (i.e., showing a particular property).
\end{itemize}

For example, in Korean, [ʃ] is a pure allophone of [s].

\begin{shortexample}[Korean secondary palatalization]
s~$\rightarrow$~~ʃ~/~\rule{1em}{.5pt}~i
\label{ex:palatalization}
\end{shortexample}

\section{Unlabeled examples}

The \texttt{unlabeledexample} environment can be used for

\begin{itemize}
\item data tables that don't require an explicit label, or
\item mathematical equations.
\end{itemize}

\citet{Zipf1949} notes a linear relationship between log word frequency $r$ and log frequency $r$. A generalized form of this relationship, shown in (\ref{ex:zipf}), is what is now known as Zipf's Law \citep[e.g.,][]{Baroni2009}.

\begin{unlabeledexample}
\label{ex:zipf}
$\displaystyle f(C, \alpha) = \frac{C}{r^\alpha}$
\end{unlabeledexample}

\begin{unlabeledexample}
$e = mc^2$
\end{unlabeledexample}

\section{Smooshed bibliographies}

Highly compact \texttt{natbib} bibliographies can be generated by using the \texttt{smooshedbib} package. Note that this is compatible with the \texttt{abbvnat} bibliography style but may not work with arbitrary styles.

\bibliography{\jobname}
\bibliographystyle{abbrvnat}
\end{document}
